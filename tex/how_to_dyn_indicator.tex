\documentclass[10pt,a4paper]{article}
\usepackage[utf8]{inputenc}

\usepackage{amsmath}
\usepackage{mathtools}
\usepackage{amsfonts}
\usepackage{amssymb}
\usepackage{amsmath}
\usepackage{siunitx}
\usepackage{physics}

\mathtoolsset{showonlyrefs,showmanualtags}

\usepackage{graphicx}
\graphicspath{{../fig2}} %Setting the graphicspath
\usepackage{subfig}
\usepackage{wrapfig}
\usepackage{sidecap}
\usepackage{booktabs}
\usepackage{hyperref}

\newtheorem{theorem}{Theorem}[section]
\newtheorem{corollary}{Corollary}[theorem]
\newtheorem{lemma}[theorem]{Lemma}

%%%%%% TESTO EFFETTIVO

\title{Algorithms used for computing the Dynamic Indicators}
\author{Carlo Emilio Montanari}

\begin{document}

\maketitle

\begin{abstract}
    In this short report, I will briefly present the various algorithms I implemented in the general Dynamic Indicator analysis. I will focus exclusively on the raw implementation, without any theoretical consideration.
\end{abstract}

\section{General notation}

I will use the following notation:

\begin{itemize}
    \item $\vb{x}_0$ is a generic 4D initial condition $(x_0, p_{x0}, y, p_{y0})$.
    \item $M^t(\vb{x})$ is the vector after the execution of $t$ map turns.
\end{itemize}


\section{Fast Lyapunov Indicator $(LI)$}

Let $\vb{x}_0$ be a generic 4D initial condition and $\vb{x}'_0 = \vb{x}_0 + \epsilon \vb{\eta}$, where $\epsilon$ is the magnitude of the displacement and $\eta$ a random unitary vector in the 4D space.

Iterate $\vb{x}_0$ and $\vb{x}'_0$ for $n$ iterations.

Finally:
\begin{equation}
LI_n(\vb{x}_0) = \frac{\log_{10} \left(\frac{(M^n(\vb{x}_0)-M^n(\vb{x}'_0))^2}{\epsilon}\right)}{n}
\end{equation}

\section{Invariant Lyapunov Error (LEI)}

Let $\vb{x}_0$ be a generic 4D initial condition and let's consider next 4 different displacements:
\begin{enumerate}
    \item $\vb{x}_1 = \vb{x} + \epsilon \vb{e}_0$
    \item $\vb{x}_2 = \vb{x} + \epsilon \vb{e}_1$
    \item $\vb{x}_3 = \vb{x} + \epsilon \vb{e}_2$
    \item $\vb{x}_4 = \vb{x} + \epsilon \vb{e}_3$
\end{enumerate}
Iterate these 5 initial conditions for $n$ turns.

Build then the following approximated tangent matrix:
\begin{equation}
    L = \begin{pmatrix}
        M^n(\vb{x}_1)_x - M^n(\vb{x}_0)_x & M^n(\vb{x}_1)_{px} - M^n(\vb{x}_0)_{px} &
        M^n(\vb{x}_1)_y - M^n(\vb{x}_0)_y & M^n(\vb{x}_1)_{py} - M^n(\vb{x}_0)_{py} \\
        M^n(\vb{x}_2)_x - M^n(\vb{x}_0)_x & M^n(\vb{x}_2)_{px} - M^n(\vb{x}_0)_{px} &
        M^n(\vb{x}_2)_y - M^n(\vb{x}_0)_y & M^n(\vb{x}_2)_{py} - M^n(\vb{x}_0)_{py} \\
        M^n(\vb{x}_3)_x - M^n(\vb{x}_0)_x & M^n(\vb{x}_3)_{px} - M^n(\vb{x}_0)_{px} &
        M^n(\vb{x}_3)_y - M^n(\vb{x}_0)_y & M^n(\vb{x}_3)_{py} - M^n(\vb{x}_0)_{py} \\
        M^n(\vb{x}_4)_x - M^n(\vb{x}_0)_x & M^n(\vb{x}_4)_{px} - M^n(\vb{x}_0)_{px} &
        M^n(\vb{x}_4)_y - M^n(\vb{x}_0)_y & M^n(\vb{x}_4)_{py} - M^n(\vb{x}_0)_{py} \\ 
    \end{pmatrix}
\end{equation}

Compute $\Sigma^2 = L^TL$

We have in the end:
\begin{equation}
    LEI_n(\vb{x_0}) = \Tr(\Sigma^2)
\end{equation}

We might be interested higher invariants of the Lyapunov error. To do so, we can use the Faddeev Leverrier recurrence

\begin{align}
    B_1 &= A \qquad &p_1 &= \Tr(B_1) \\
    B_2 &= A(B_1 - p_1 I) \qquad &p_2 &= \frac{1}{2} \Tr(B_2) \\
    \vdots\\
    B_k &= A(B_{k-1} - p_{k-1} I) \qquad &p_k &= \frac{1}{k} \Tr(B_k) \\
\end{align}
(we notice that $\Tr(\Sigma^2)$ is in fact $p_1$).

And we can finally consider:
\begin{equation}
    LEI^k_n(\vb{x_0}) = (-1)^{k+1} p_k \qquad A=\Sigma^2_Ln = L^TL
\end{equation}

\section{Reversibility Error (RE)}

Given $\vb{x_0}$, we define the following

\begin{equation}
    RE_n(\vb{x}_0) = (M^{-1})^n(M^{n}(\vb{x}_x))
\end{equation}

An arbitrary random kick of given average and standard deviation can be considered for each and every map step, or only part of the steps (e.g.\ kick the particles only in the forward tracking).

\section{Invariant Reversibility Error (REI)}

We consider the initial condition $\vb{x}_0$, and we execute the same forward-backward tracking we did for RE. We end up with a displaced vector $\vb{x}_1$.

We then compose the covariance matrix
\begin{equation}
    
\end{equation} 

\section{SALI}

\section{GALI}

\section{MEGNO}

\section{Frequency Map}

\end{document}